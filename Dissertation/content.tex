\chapter{About this project}
\section{Abstract}
This project is a Mobile Application for the Android Operating System. The programming Language used in Kotlin. The development Environment that the application is written in is Intellij IDEA Ultimate. \\
The app is a sport based application that allows the user to find all golf courses in Ireland. The user will be given the weather forecast over a 3, 7 or 10 day period depending on their preference. The weather forecast will be specific to what golf course that the user wants to play at. It will give the user back the best hour and day to play on for that period chosen. \\ 
The application will allow the user to access a MongoDB database to store and fetch personalized data. The stored data is your score for each hole and what golf club you used for that hole on that golf course and will generate a weekly/monthly report for how you performed on each golf course that you used during that time period.\\
\section{Authors}
For this Project, the authors are Robert Donnelly, Evan Greaney and Steven Joyce. All three authors are 4th Software development students, attending GMIT.

\chapter{Introduction}

For our final year project we were tasked with designing , developing and deploying a project either individually or as a group . We decided to work together as a group in order to achieve our goals of creating an app that would assist as a companion to players for one sport with the hope of adding in more sports in the future. The sport we decided to focus on was the game of golf, in which the app would give players optimal playing days and times based on weather conditions as well as their preferred conditions for playing golf. The application would also allow players to save their score data for that day which can be accessed in a report or table.

\section{Objectives}
\begin{itemize}
    \item Find a new programming language to learn independently
    \item Agree on an idea of an Application to develop
    \item Find a Methodology to best implement for our idea
    \item Implement the methodology to create a project plan
    \item Find an architecture structure suitable to use
    \item Utilize an environment for stable version control
    \item Establish a well structured base for the project
    \item Create a high quality front-end
    \item Find and deploy an API which our app can utilize
    \item Create a server service for the Application.
    \item Allow for user data to be protected and appropriately stored
    \item Use of a Location service to find user's desired location
    \item Implement a login service which allows user to access their protected data
    \item Design the Apps software to successfully implement the apps intended functionality 
    \item Deploy our methodologies style of testing to ensure quality standards are met
    
\end{itemize}
\section{Chapters}
\subsection{Methodology}
This chapter covers our implementation of the Agile Methodology used to collaborate as a team to structure our project schedule to take an incremental and iterative approach to the development of our Mobile Application.
\subsection{Technology Review}
In this Area of this Dissertation, it will cover all technologies used to develop this application,to design this application and will give the reader an understanding of why these technologies were used.
\subsection{System Design}
The chapter of System Design highlights how this Mobile Application was developed using the technologies outlined in the Technology Review chapter. It will also outline and illustrate the System design that was imagined and conceived.
\subsection{System Evaluation}
The System Evaluation section of this Dissertation will cover the testing stage of the Application,where in the finalised Application will be compared to the initial objectives of the Application set out by the Introduction.It will outline the faults and limitations of the technologies used to design and develop the project along with concepts for future advancement of the Application.  

\subsection{GitHub Repository}
Link to GitHub Repository={\url{https://github.com/stevenJoyce/4thYearGroupProject}}
\newline
\begin{enumerate}
    \item \textbf{App Folder:} 
    \newline This folder contains the Files necessary to build the Application
    \item \textbf{Dissertation Folder:}
    \newline This is where the Dissertation can be run from Overleaf 
    \item \textbf{Dissertation PDF:}
    \newline This is where the final version of the Dissertation can be viewed from
    \item \textbf{ReadMe:}
    \newline Gives a brief overview of what is contained within the GitHub Repository
    \item \textbf{Presentation of Initial Concept:}
    \newline This is a PowerPoint presentation designed highlight the initial concepts and prototypes for the project
    \item \textbf{Project Gantt:}
    \newline This is an excel file that displays the implementation of Agile Methodologies to design sprints of development
    \item \textbf{.Ideas Folder:}
    \newline This folder is where we ran and debugged our Application from in the IntelliJ IDE
    \item \textbf{Gradle Folders:}
    \newline Using this folder, this is how the Application is configured using IntelliJ
\end{enumerate}


The introduction should be about three to five pages long.
Make sure you use references

This project is an Android Application built using the programming Language. The app will allow the user to 


\chapter{Methodology}
\section{Approach}
Agile
\section{Testing}
emulation , terminal, test cases
\section{collaborative platforms}
GitHub, Discord, Teams
\section{Technology choices}
intellij IDEA Ultimate, Android Studio via Kotlin, Mongo Realm, JustInMind
\section{Algorithm's}
our own developed rating system.

About one to two pages.
Describe the way you went about your project:
\begin{itemize}
\item Agile / incremental and iterative approach to development. Planning, meetings.
\item What about validation and testing? Junit or some other framework.
\item If team based, did you use GitHub during the development process.
\item Selection criteria for algorithms, languages, platforms and technolo-gies.
\end{itemize}


\chapter{Technology Review}
\section{Kotlin}
\section{Intellij IDEA's}
\section{API's used}
potential for subsections
\section{MongoDB}
\subsection{Cluster}
\subsection{Realm}
\section{GitHub}
\section{Comunication Application's}
\subsection{Discord}
\subsection{Teams}

About seven to ten pages.
\begin{itemize}
\item Describe each of the technologies you used at a conceptual level. Standards, Database Model (e.g. MongoDB, CouchDB), XMl, WSDL, JSON, JAXP.
\item Use references (IEEE format, e.g. [1]), Books, Papers, URLs (timestamp) – sources should be authoritative. 
\end{itemize}

\section{XML}
Here's some nicely formatted XML:
\begin{minted}{xml}
<this>
  <looks lookswhat="good">
    Good
  </looks>
</this>
\end{minted}
\chapter{System Design}
As many pages as needed.
\begin{itemize}
\item Architecture, UML etc. An overview of the different components of the system. Diagrams etc… Screen shots etc.
\end{itemize}
\begin{table}[h]
  \centering
  \begin{tabular}{x{2cm}p{3cm}}
    \toprule \\
    Column 1 & Column 2 \\
    \midrule \\
    Rows 2.1 & Row 2.2 \\
    \bottomrule
  \end{tabular}
  \caption{A table.}
  \label{table:mytable}
\end{table}
\chapter{System Evaluation}
As many pages as needed.
\section{Unit Testing}
\section{Application Deployment}
\section{Application Features}
\section{Objectives Overview}
\section{Issues, limitations}
\section{Future of Application}
\begin{itemize}
\item Prove that your software is robust. How? Testing etc. 
\item Use performance benchmarks (space and time) if algorithmic.
\item Measure the outcomes / outputs of your system / software against the objectives from the Introduction.
\item Highlight any limitations or opportunities in your approach or technologies used.
\end{itemize}
\chapter{Conclusion}
About three pages.
\begin{itemize}
\item Briefly summarise your context and objectives (a few lines).
\item Highlight your findings from the evaluation section / chapter and any opportunities identified.
\end{itemize}
\chapter{References}
\chapter{Appendices}