\chapter{About this project}
\section{Abstract}
This project is a Mobile Application for the Android Operating System. The programming Language used in Kotlin. The development Environment that the application is written in is Intellij IDEA Ultimate. \\
The app is a sport based application that allows the user to find all golf courses in Ireland. The user will be given the weather forecast over a 3, 7 or 10 day period depending on their preference. The weather forecast will be specific to what golf course that the user wants to play at. It will give the user back the best hour and day to play on for that period chosen. \\ 
The application will allow the user to access a MongoDB database to store and fetch personalized data. The stored data is your score for each hole and what golf club you used for that hole on that golf course and will generate a weekly/monthly report for how you performed on each golf course that you used during that time period.\\
\section{Authors}
For this Project, the authors are Robert Donnelly, Evan Greaney and Steven Joyce. All three authors are 4th Software development students, attending GMIT.
\chapter{Introduction}

For our final year project we were tasked with designing , developing and deploying a project either individually or as a group . We decided to work together as a group in order to achieve our goals of creating an app that would assist as a companion to players for one sport with the hope of adding in more sports in the future. The sport we decided to focus on was the game of golf, in which the app would give players optimal playing days and times based on weather conditions as well as their preferred conditions for playing golf. The application would also allow players to save their score data for that day which can be accessed in a report or table.

\section{Objectives}
\begin{itemize}
    \item Find a new programming language to learn independently
    \item Agree on an idea of an Application to develop
    \item Find a Methodology to best implement for our idea
    \item Implement the methodology to create a project plan
    \item Find an architecture structure suitable to use
    \item Utilize an environment for stable version control
    \item Establish a well structured base for the project
    \item Create a high quality front-end
    \item Find and deploy an API which our app can utilize
    \item Create a server service for the Application.
    \item Allow for user data to be protected and appropriately stored
    \item Use of a Location service to find user's desired location
    \item Implement a login service which allows user to access their protected data
    \item Design the Apps software to successfully implement the apps intended functionality 
    \item Deploy our methodologies style of testing to ensure quality standards are met
    
\end{itemize}
\section{Chapters}
\subsection{Methodology}
This chapter covers our implementation of the Agile Methodology used to collaborate as a team to structure our project schedule to take an incremental and iterative approach to the development of our Mobile Application.
\subsection{Technology Review}
In this Area of this Dissertation, it will cover all technologies used to develop this application,to design this application and will give the reader an understanding of why these technologies were used.
\subsection{System Design}
The chapter of System Design highlights how this Mobile Application was developed using the technologies outlined in the Technology Review chapter. It will also outline and illustrate the System design that was imagined and conceived.
\subsection{System Evaluation}
The System Evaluation section of this Dissertation will cover the testing stage of the Application,where in the finalised Application will be compared to the initial objectives of the Application set out by the Introduction.It will outline the faults and limitations of the technologies used to design and develop the project along with concepts for future advancement of the Application.  

\subsection{GitHub Repository}
Link to GitHub Repository={\url{https://github.com/stevenJoyce/4thYearGroupProject}}
\newline
\begin{enumerate}
    \item \textbf{App Folder:} 
    \newline This folder contains the Files necessary to build the Application
    \item \textbf{Dissertation Folder:}
    \newline This is where the Dissertation can be run from Overleaf 
    \item \textbf{Dissertation PDF:}
    \newline This is where the final version of the Dissertation can be viewed from
    \item \textbf{ReadMe:}
    \newline Gives a brief overview of what is contained within the GitHub Repository
    \item \textbf{Presentation of Initial Concept:}
    \newline This is a PowerPoint presentation designed highlight the initial concepts and prototypes for the project
    \item \textbf{Project Gantt:}
    \newline This is an excel file that displays the implementation of Agile Methodologies to design sprints of development
    \item \textbf{.Ideas Folder:}
    \newline This folder is where we ran and debugged our Application from in the IntelliJ IDE
    \item \textbf{Gradle Folders:}
    \newline Using this folder, this is how the Application is configured using IntelliJ
    \item \textbf{Wireframe Prototype:}
    \newline In this folder we have a Justinmind prototype outlining the design of the application.
    \item \textbf{Manual}
    \newline The guide to installing software needed to run the application.
\end{enumerate}


The introduction should be about three to five pages long.
Make sure you use references

This project is an Android Application built using the programming Language. The app will allow the user to 
\chapter{Methodology}
\section{Approach}
The Agile methodology is based on a iterative development approach that allows requirements and solutions to change throughout the development of the Application. It relys on constant communication between members of the team to successfully deliver a final product which has the capacity to evolve. Teams will undergo sprints which splits up a feature into smaller parts that is then given to a member of the team to work on in a short time period which makes the feature more manageable to deliver.
\newline
\newline
To implement this approach our team organized weekly meetings where we discussed our work's success or issues and how far we have gotten with our given sprint . This allowed us to collaborate as a team when issues occurred allowing us to keep on schedule. Our schedule was developed using a gantt chart, this was the basis for all our sprints . Every step was assigned as a sprint which was assigned to each member to carry out. Each sprint had a test case which had to pass for us to proceed with development.
\section{Testing}
After each sprint a set of tests were required to pass. If a test case failed to pass in any capacity we could not proceed with the next sprint. To test our application we used  the Android SDK emulator alongside intellij IDEA.
\newline
The emulator we used was run on Android 9.0 (pie). The emulator simulated a Google Pixel 3A mobile phone. 
(emulation , terminal, test cases)
\section{Collaborative Platforms}
\subsection{Version Control}
The platform all team members were most familiar with was GitHub, it allows members to access project code versions and commit their contributions to one repository. Every team member has access to view and modify the contents of the repository. All changes are logged and any commit can be accessed at all times , this helps with problem solving as any issues that occur can be called back on to older commits that were functional if issues occur with the latest commit.

\subsection{Communication Tools}
\subsubsection{Microsoft Teams}
Microsoft Teams was used primarily to communicate with our supervisor. We chose this as it is linked to our student accounts and our supervisor requested us to use it as a way to communicate with them. Teams is a good platform for video calls as it can be accessed in multiple platforms such as an Internet browser, Mobile/Desktop Application . It allows users to share files and can sync to your google drive account(file hosting service).and use a calendar feature allowing users to organize meetings.
\newline
However it is not strongly suited for messaging as its forum feature can become cluttered and tedious to navigate,
This is one of the main reasons why we decided to use Discord as our primary form of communication as a team.
\subsubsection{Discord} Discord is a communication Application which has increased in popularity in recent years, especially considering the current pandemic. It runs on Desktop, Mobile and browser platforms. This allowed us to always be in contact with each other if one team members device was running into issues.  
\newline
Discord allowed us to set up separate channels to sort our messages to relate to specific aspects of the Project. We could co ordinate and plan different parts of the project in the one place with no clutter, Each aspect of the project had a dedicated channel so that unrelated messages would not be spam our feed. For example we had an "ideas" channel where we would jot down random ideas or inspiration for new features of the application, these messages would be separate from our "session-planning" channel which would strictly contain messages related to our sprint meetings .Due to this feature it became our primary platform to communicate.
\newline 
\subsubsection{Whats App}
We used Whats App on our mobile devices to organize our sprint meetings as a team. If for any reason a meeting needed to be held on short notice Whats App was the most efficient way of coordinating a set time.
\subsection{Literature Software}
For the Development of this Dissertation we used Overleaf, we found this Latex text editor the effective way to write up due our previous experience. The Documentation Overleaf provides is well structured which helped us with this aspect of the project. After each person wrote a part of the dissertation it was committed to GitHub, which helped with making sure every member of the team had the appropriate version of the Dissertation.
(GitHub, Discord, Teams)
\section{Technology choices}
\subsection{Kotlin}
As a team we wanted to learn a new programming language as a challenge for our project and to increase our skill set as Software Developers. One of our lecturer's during one of our online classes from this module mentioned the Kotlin language. This led us to research this language and understand what it could be used for. We saw that Kotlin was an up and coming language that was primarily used for mobile Applications in Android as of September 2020 with hopes of iOS support being implemented in the near future.
\subsection{Intellij IDEA}
At the start of the college year we were introduced to Intellij in our Distributed Systems module. We soon realised that JetBrains who are the founders of Intellij are also the designers of the Kotlin programming language. With this information we decided to use this as our chosen development environment.
\newline
We soon realised that we had the wrong version of Intellij installed on our devices. We needed to have the Intellij IDEA Ultimate to be able to access a Mongo Database. We got this free with the GitHub Student Developer pack. This was an issue that was unforeseen and time consuming but easily rectified.
\newline 
With JetBrains owning both Kotlin and Intellij writing the code was seamless and allowed testing to constantly occur. Intellij allowed the use of an Android mobile phone emulator to run these tests.
\subsection{MongoDB}
MongoDB is a NoSQL database service which allows a user to store data from an application for future use. The data stored can be queried in searches which outputs specific results. These result can be processed by the application and read out to the user. We chose to use MongoDB as our server because of our previous experience of developing applications with it as the back end storage. 
\newline
\newline
To use a MongoDB cluster we had to create a Mongo Realm application. This is where the user data can be read into the Kotlin application. 
This can also send data to the cluster through this Realm application for storage. Realm reads in a cluster which it has permission to access the stored data through a partition key and clusters data to the Kotlin application.
\subsection{JustInMind}
In order to conceptualize and visualise our applications front end layout we had to begin with prototyping our design. While researching design tools for mobile application development, we came across a prototyping tool called JustInMind.
\newline
It is a high level prototyping tool used to create high quality wire frames that allow developers to visualise their product in real time before finalizing the design of the mobile application. It assisted us in figuring out how we want the user to navigate the application.

\section{Algorithm's}
For our own rating system we are using JFuzzyLogic to output the best time and day for the user to play golf, determined by metrics based upon the user's input. \newline
"JFuzzyLogic is an extension of Boolean logic by Lotfi Zadeh in 1965 based on the mathematical theory of fuzzy sets, which is a generalization of classical set theory." - Quote from Franck Dernoncourt.

\chapter{Technology Review}
\section{Kotlin}
We found Kotlin to be less verbose, easier to read as a programming language compared such as Java, C. It is a high-level language that can be used to produce mobile applications. It's coding style is similar to python but with more features and ways to implement it.  \newline \newline
This is due to Kotlin not needing any end of line syntax to allow the interpreter move on to a new line. - Subject to change
\newline \newline
The Kotlin language is a very good interpreter of other languages which helps when trying to write code without having a great grasp of the language. This is very important when you are 1st learning the language or when you cannot find a way to write a method in Kotlin but can in Java. 
\newline
Despite both Kotlin and Java being a native language for writing Android applications, Kotlin is seen by many as a cleaner and concise way to write code for mobile than Java. The amount of lines of code that needs to be written can be up. 

\section{Intellij IDEA's}
The development environment provided by Intellij allows for the development of several different types of programs including Mobile Applications. The IDE contains a built in android development environment with Kotlin being the default language. This is due to the development lead designer being Roman Elizarov.
\newline
The Android SDK can be modified within the IDE, which allows a developer to seamlessly link the project to an android version to test on an emulator. That feature is a great way to debug a mobile application efficiently and in real-time.
\section{API's used}
potential for subsections

\subsection{Open Weather API}
Open Weather is an API that allows the user to retrieve weather data from any location using its NWP (Numerical Weather Prediction) Model. This API has been designed to be used universally on any Language to gather data. It gathers this data using a "proprietary convolutional neural network that collects and processes wide range of data sources to cover any location and consider the local nuances of climate." (Quote from "https://openweathermap.org/guide" - Openweather NWP model)
\subsubsection{Features}

\subsection{Geocoding API}
In order to retrieve data with the Open Weather API, we needed the longitude and latitude of a chosen golf course selected by the user. With the help of the Geocoding API it gives the ability to to convert street name data into longitude and latitude values for the Open Weather API to search weather data based on that location.

\subsubsection{Features}

\section{MongoDB}
To utilise the MongoDB database,we had to use two of its three main core features, these two features are Atlas and Realm,Atlas and Realm are integrated in a way that allows for seamless connections for the Android Application
\subsection{Atlas}
\subsection{Realm}
\section{GitHub}
\section{Comunication Application's}
\subsection{Discord}
\subsection{Teams}

About seven to ten pages.
\begin{itemize}
\item Describe each of the technologies you used at a conceptual level. Standards, Database Model (e.g. MongoDB, CouchDB), XMl, WSDL, JSON, JAXP.
\item Use references (IEEE format, e.g. [1]), Books, Papers, URLs (timestamp) – sources should be authoritative. 
\end{itemize}

\chapter{System Design}
As many pages as needed.
\begin{itemize}
\item Architecture, UML etc. An overview of the different components of the system. Diagrams etc… Screen shots etc.
\end{itemize}
\begin{table}[h]
  \centering
  \begin{tabular}{x{2cm}p{3cm}}
    \toprule \\
    Column 1 & Column 2 \\
    \midrule \\
    Rows 2.1 & Row 2.2 \\
    \bottomrule
  \end{tabular}
  \caption{A table.}
  \label{table:mytable}
\end{table}
\chapter{System Evaluation}
As many pages as needed.
\section{Unit Testing}
\section{Application Deployment}
\section{Application Features}
\section{Objectives Overview}
\section{Issues, limitations}
\section{Future of Application}
\begin{itemize}
\item Prove that your software is robust. How? Testing etc. 
\item Use performance benchmarks (space and time) if algorithmic.
\item Measure the outcomes / outputs of your system / software against the objectives from the Introduction.
\item Highlight any limitations or opportunities in your approach or technologies used.
\end{itemize}
\chapter{Conclusion}
About three pages.
\begin{itemize}
\item Briefly summarise your context and objectives (a few lines).
\item Highlight your findings from the evaluation section / chapter and any opportunities identified.
\end{itemize}
\chapter{References}
\chapter{Appendices}
